%!TEX root = thesis_main.tex

\chapter{Conclusion}\label{ch:conclusion}
% I did it I did it... Now give me my piece of paper! 

Cycloidal actuators provide the capability for very large reductions in a compact package with high torque capability and are generally compared directly with harmonic drives. Compared to a harmonic drive, a cycloidal actuator can provide a tx increase in specific torque (torque/mass) over a harmonic drive. These benefits are contrasted with known deficiencies of higher torque ripple and backlash than a harmonic drive and must be traded in the actuator design. 

While the benefits of cycloidal actuators are clear, and the theoretical analysis of single-stage cycloids are well documented in the literature, there is a gap in the literature with regards to actual in-use test data, specifically for efficiency and lifetime. Through this thesis, a single-stage cycloid with a 59:1 reduction utilizing the compact pin in housing design style was run through a series of efficiency and lifetime tests. Through the 159k output revolutions over the course of 300+ hours of testing, peak efficiencies of 81\% were achieved and no appreciable loss of efficiency over the testing occured. Additionally, it was demonstrated that a substantial break in period of 10+ hours may be necessary before steady state efficiencies are achieved. The compact single-stage cycloid tested had a 2x increase in specific torque over a harmonic drive rated for the same loads and speeds and achieved slightly higher efficiencies than published in the harmonic drive data sheets. If backlash and torque-ripple are acceptable in design, a compact cycloid is a valid design choice for a given system. 
% mention the relative motion analysis

A new addition to the literature in the last few years is the concept of a compact two-stage cycloid design in which the motion of the first stage cycloid plate is used to directly drive the motion of the second stage cycloid plate. The second stage plate drives the second stage pins to achieve very large potential reductions. The kinematics for this design have been presented in the literature with brief analysis of the loads expected on the cycloid plates. 

A two-stage cycloid was designed and tested using these kinematic equations, and a gap was identified in the analysis of these two-stage systems, resulting in large inefficiencies of the design. Therefore, analysis was done to study the interaction between the forces and velocities at the lobe-pin interaction and predict the efficiency losses associated with this. It was demonstrated that the losses associated with the given design for a compact pin in housing design result in best case efficencies of ~36\% suggesting that high ratio designs must utilize rolling elements for these interactions to achieve acceptable efficiencies. Additionally, comparrisons were done to give guidelines for designers. A two-stage cycloid with a negative ratio arrangement results in increased efficiency for a given coefficient of friction for the pin to housing interaction. A design utilizing more pins also results in a higher efficency. Finally, a lower ratio with similar outputs can result in higher efficiencies. 


