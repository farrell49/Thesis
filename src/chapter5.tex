%!TEX root = thesis_main.tex

\chapter{Conclusion}\label{ch:conclusion}
% I did it I did it... Now give me my piece of paper! 

Cycloidal actuators provide the opportunity for very large reductions in a compact package with high torque capability. Cycloidal actuators are generally compared directly with harmonic drives. Compared to a harmonic drive, a cycloidal actuator can provide a 2x increase in specific torque (torque/mass). These benefits are contrasted with known deficiencies of higher torque ripple and backlash than a harmonic drive and must be traded in the actuator design. 

While the benefits of cycloidal actuators are clear and the theoretical analysis of single-stage cycloids are well documented in the literature, there are two identified gaps in the literature. The first gap is a lack of actual in-use test data, specifically for efficiency and lifetime. The second is a closed form solution of the velocity of the lobe to pin interaction and the losses associated with this interaction. 

Through this thesis, a single-stage cycloid with a 59:1 reduction utilizing the compact pin-in-housing design style was run through a series of efficiency and lifetime tests. Through the 129,000 output revolutions over the course of 300+ hours of testing, peak efficiencies of 81\% were achieved with no appreciable loss of efficiency over the testing period. Additionally, it was demonstrated that a substantial break-in period of 10+ hours may be necessary before steady-state efficiencies are achieved. The compact single-stage cycloid tested had a 2x increase in specific torque over a harmonic drive rated for the same loads and speeds and achieved slightly higher efficiencies than published in the harmonic drive data sheets. If backlash and torque-ripple are acceptable in design, a compact cycloid is a valid design choice. Additionally, the closed form equations for the relative velocity between the cycloid plate and housing pins/rollers were developed and demonstrated. An analysis was conducted to compare to the tested cycloid. This analysis supported the results of the efficiency testing of the single-stage design.
% mention the relative motion analysis

A new addition to the literature within the last few years is the concept of a compact two-stage cycloid design in which the motion of the first stage cycloid plate is used to directly drive the motion of the second stage cycloid plate. The second stage plate drives the second stage pins to achieve very large potential reductions. The kinematics for this design have been presented in the literature with brief analyses of the loads expected on the cycloid plates. 

In this thesis a two-stage cycloid was designed and tested using these kinematic equations, and a gap was identified in the analysis of these two-stage systems, resulting in large inefficiencies of the design. Therefore, analysis was done to study the interaction between the forces and velocities at the lobe-pin interaction and predict the efficiency losses associated with this interaction. This work demonstrates that the losses associated with the as-built design for a compact pin-in-housing design result in best-case efficiencies of ~36\%, suggesting that high ratio designs must utilize rolling elements for these interactions to achieve acceptable efficiencies. Additionally, comparisons were done to give guidelines for designers. A two-stage cycloid with a negative ratio arrangement results in increased efficiency for a given coefficient of friction for the pin to housing interaction. A design utilizing more pins also results in a higher efficiency. Finally, a lower ratio with similar outputs can result in higher efficiencies. 

Overall, this work has demonstrated that a single-stage cycloid is a valid design choice in robotic actuators. If backlash and torque-ripple are acceptable, a single-stage, pin-in-housing design can achieve higher efficiencies than a harmonic drive with a reasonable expected lifetime and half the mass. In addition, this work contributes additional necessary design equations for the sizing and design of a two-stage cycloid system. Further experimental testing will need to be done to validate the losses of a bearing-in-housing design of this system. 
