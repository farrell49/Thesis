% 7 Questions 
% 1. Focus/Problem to be Solved 
% 2. Importance - why is this important to people?
% 3. Methods - how did we test this? 
% 4. Context - talk about them in context with relevant literature and such - why do people care? 
% 5. Results - what did we show 
% 6. Unique Contribution - what is unique and why do people care? 
% 7. Possible Applications 

% 1. 
% The problem to be solved is two-fold. First, how well do single-stage cycloids of this compact pin in housing design actually work? Very little literature exists on it. The second problem is how well do two-stage cycloids work? Are they a good use of a high reduction in a small package? 
% 2.
% This is important because it adds another high reduction capability if you relax some requirements, you can get a 2x increase in specific torque with a single stage, and a huge ratio and torque capacity in a small package with a two stage if they were to actually work well. 
% 3.
% We built two test articles, a single stage, and a two-stage based on the relevant literature. We tested the single-stage for 300+ hours and 129k revolutions to determine true efficiency, run-in, and give an indication to lifetime. We ran the two-stage briefly that indicated gaps in understanding of the losses of the system, since kinematic analysis and brief force analysis are the only things laid out. So a new set of equations were developed to show loss trends in these cycloids and identify the reason of the poor efficiency of the as-built two-stage. 
% 4. 
% 
% 5. Results 
% The compact single stage cycloid showed efficiencies similar, and slight above those published by Harmonic drive. In addition, a notable burn in time was required before steady state performance was reached and through the 300 hours of running, no appreciable loss of efficiency was seen. Also, the efficiency was not flat with torque.  For the two-stage cycloids, the loss calculations show that, to achieve any reasonable efficiency, the lobe to pin interactions should be made from rolling elements, and the theoretical calculations fall in the same region as the actual system performed, substantiating those claims. 

% 6. Unique Contribution 
% To this point, very little testing has been done on a single-stage cycloid, only about 80 minutes. This long duration testing shows that high efficiencies (80%), similar to harmonic can be achieved for substantial amounts of time. This validates this drive as a possible replacement for harmonic in cases where backlash is acceptable. Additionally, very little aside from basic kinematic analysis of the two-stage concept has been performed. This analysis gives design recommendations that were not previously published, allowing designers to design better two-stages. 

% 7. Possible Applications 
% This shows that single-stage cycloids of the compact style could be used to replace harmonic drives in cases where backlash is acceptable. Additionally, the high reuduction drives, if made with rolling elements, could potentially have pretty high efficiencies and could provide a very nice high reduction compact package. 

Many robotic applications demand compact, high reduction drives for their actuators. To date, the common actuator used is a harmonic drives for this application. However, cycloidal drives could be considered for these applications as they provide high reductions in a compact package, are highly customizable, and can be easily manufactured compared to a harmonic drive. Single-stage cycloids have been well analyzed in the literature, but not well tested, with the most published testing of 80 minutes. In this work, a single-stage cycloid was built and run for 300+ hours and 129,000 output revolutions to determine in-use efficiency and lifetime. This testing demonstrated that the compact, pin-in-housing designs can achieve efficiencies near and above that of a comparable harmonic drive with a peak of 81\% as well as a 2x increase in specific torque. A substantial burn-in time was noticed of about 8 hours, and the efficiency did not degrade appreciably over the course of the testing. A new cycloid design for a two-stage cycloid has been recently proposed and the basic kinematic analysis conducted. A test article for this two-stage design was constructed and tested. This identified gaps in the literature regarding the losses associated with the lobe to pin interactions. This work develops the mathematics necessary to characterize these losses in theory and they are compared to the tested actuator. The actuator losses nearly match the predicted losses of the two-stage system. This work presents many additional design equations for a two-stage cycloid system, the primary result suggesting that a two-stage system must be built with rolling elements in the housing to achieve satisfactory (above 50\%) efficiencies. This work demonstrates that single-stage cycloids are a viable replacement for harmonic drive in high reduction applications where backlash is allowed and demonstrate additional design equations necessary for a two-stage cycloid design.

\todo[inline]{I present "single-stage intro results" then "two-stage intro results." Would it flow better to intro both, then results both?}

\todo[author=LF, inline]{Fix the patent refs}