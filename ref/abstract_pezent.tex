\thispagestyle{empty}
\begin{abstract}


Robotic devices have been clinically verified for use in long duration and high intensity rehabilitation needed for motor recovery after neurological injury. Targeted and coordinated hand and wrist therapy, often overlooked in rehabilitation robotics, is required to regain the ability to perform activities of daily living. To this end, a new coupled hand-wrist exoskeleton has been designed. This thesis details the design of the wrist module and several human-related considerations made to maximize its potential as a coordinated hand-wrist device. The serial wrist mechanism has been engineered to facilitate donning and doffing for impaired subjects and to insure compatibility with the hand module in virtual and assisted grasping tasks. Several other practical requirements have also been addressed, including device ergonomics, clinician-friendliness, and ambidextrous reconfigurability. The wrist module's capabilities as a rehabilitation training device are quantified experimentally in terms of functional workspace and dynamic properties. Finally, the device is validated as an rehabilitation assessment tool by considering its impact on commonly used assessment metrics. The presented wrist module's performance and operational considerations support its use in a wide range of future clinical investigations.

%Specifically, the device possesses favorable performance in terms of range of motion, torque output, friction, and closed-loop position bandwidth when compared with existing devices.

\end{abstract}

